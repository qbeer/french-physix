\documentclass[12pt, a4paper]{article}
\usepackage[french]{babel}
\usepackage[T1]{fontenc}
\usepackage[utf8]{inputenc}
\usepackage{amsmath}
\usepackage{float}

\author{}
\title{Premier devoir de la mécanique quantique}
\date{\today}

\begin{document}
\maketitle
\section{Exercise}
\par Avec  l'utilisation d'équation Schrödinger il faut que vous prouviez la théorem d'Ehrenfest:
\begin{equation*}
\frac{d}{dt}\big \langle \hat{p} \big \rangle = \bigg \langle \frac{\partial \hat{V}(x)}{\partial x} \bigg \rangle
\end{equation*}
\section{Exercise}
\par Une balle en plastique bondis sur la terre dans la gravité homogène de la terre. Si on utilise
la Bohr-Sommerfeld assumption quel est la taille minimum a > 0 pour faire bondir la balle?
\section{Exercise}
\par Les opérateur Hamilton $-\frac{\hbar^{2}}{2m}\frac{\partial^{2}}{\partial x^{2}} + V(x)$, momentum
$-i\hbar\frac{\partial}{\partial x}$, l'opérateur de la translation $\hat{T}$ qui $\hat{T}(a) \Phi(x)= \Phi(x-a)$ et 
finalement l'opérateur de le réflection $\hat{R}$ qui $\hat{R}\Phi(x) = \Phi(-x)$. Quelles sont les valeurs des commutateures?
\begin{itemize}
    \item $\Big[\hat{H}, \hat{p}\Big]$
    \item $\Big[\hat{H}, \hat{T}(a)\Big]$
    \item $\Big[\hat{H}, \hat{R}\Big]$
    \item $\Big[\hat{T}(a), \hat{R}\Big]$
\end{itemize}
\end{document}