\documentclass[12pt, a4paper]{article}
\usepackage[french]{babel}
\usepackage[T1]{fontenc}
\usepackage[utf8]{inputenc}
\usepackage{amsmath}
\usepackage{float}
\usepackage{enumitem}
\usepackage{physics}

\author{}
\title{Troisième devoir de la mécanique quantique}
\date{\today}
\begin{document}
\maketitle
\section{Exercice}
\par Obtenez les niveaux d'énergie:
\begin{equation*}
V(x) := \begin{cases}
	+\infty & \text{0 < $x$} \\
	\frac{1}{2}m\omega^{2}x^{2} & \text{ $x \geq 0$} \\
\end{cases}
\end{equation*}
dans la vallée de la potentielle.
\begin{equation*}
\section{Exercice}
Le $CO_{2}$ est une molécule linéaire qui peut devenir un ion négatif avec l'absorption d'un électron. Pour le modèle de cette molécule
il faut assumer que l'énergie du électron sans interaction soit $E_{O}$ si on le mettre sur un des atoms oxigens et elle soit $E_{C}$ si on
le mettre sur le carbon. Les états d'électron sont $\ket{D}$ (dans l'oxygène dans le droit) $\ket{M}$ (dans le carbon au milieu) et finalement $\ket{G}$
(dans l'oxygène à gauche). Quand même les propres énergies du system sont differents de $E_{C}, E_{O}$ en raison d'effet tunnel, pour ça 
il faut que on utillise une correction $\Delta$. Donc le Hamilton opérateur du system

 est le suivante:
\begin{equation*}
\hat{H} = 
\begin{pmatrix}
E_{O} & \Delta & 0 \\
\Delta & E_{C} & \Delta \\
0 & \Delta & E_{O}
\end{pmatrix}
\end{equation*}
\begin{enumerate}[label=\alph*)]
\item Quelles sont les propres énergies du system?
\item Obtenez les vecteurs propres du system. Pour ça, il faut assumer que $E_{O} = E_{C}$. 
\end{enumerate}
\section{Exercice}
\par Le Hamilton operateur est $\hat{H} = \frac{\hat{L}^{2}_{z}}{2\Theta}$. L'état est (t = 0):
\begin{equation*}
\psi(\varphi, t = 0) = A cos^{2}\varphi
\end{equation*}
Il faut normaliser la fonctionne et résoudre la value de A pour t > 0.
\section{Exercice}
\par La fonctionne qui corresponde á un état
\begin{equation*}
\psi(x,y,z) = \frac{2(x^{2} - y^{2})}{f(x^{2}+y^{2}+z^{2})}
\end{equation*}
\par Dans quelle valeur propre on peut trouver l'électron et qulle et la probabilité qui y corresponde?
\end{document}