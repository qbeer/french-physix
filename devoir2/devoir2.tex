\documentclass[12pt, a4paper]{article}
\usepackage[french]{babel}
\usepackage[T1]{fontenc}
\usepackage[utf8]{inputenc}
\usepackage{amsmath}
\usepackage{float}
\usepackage{enumitem}

\author{}
\title{Deuxième devoir de la mécanique quantique}
\date{\today}
\begin{document}
\maketitle
\section{Exercise}
\par La potantielle est donnée comme:
\begin{equation*}
V(x) := \begin{cases}
	0 & \text{0 < $x$ < $a$} \\
	\infty & \text{autrement} \\
\end{cases}
\end{equation*}
il y a une particule dans l'état
\begin{equation*}
\Psi(x,0) = A\cdot[\Psi_{1}(x) + \Psi_{2}(x)]
\end{equation*}
où $\Psi_{1}$ et $\Psi_{2}$ font partie de la série suivante:
\begin{equation*}
\Psi_{n}(x) = \sqrt{\frac{2}{a}}\sin({\frac{n\pi}{a}x})
\end{equation*}
\begin{enumerate}[label=\alph*)]
\item Normalisez l'état au temps t = 0! Que est le paramètre A?
\item Calculez le temps développement de la fonctionne $\Psi(x,0)$ et la formule $|\Psi(x,t)^{2}|$! ( Aide: utilisez la formule d'Euler à
résoudre: $e^{i\varphi} = cos(\varphi) + isin(\varphi)$ et choissisez le paramètre $\omega = \frac{\pi h}{4ma^{2}}$ )
\item Quelle est la valeur prévue de l'operateur $\hat{x}$? Quelle est la fréquence de son oscillation et l'amplitude? ( Ce peut pas être plus que
$\frac{a}{2}$ )
\item Quelle est la valeur prévue de l'operateur $\hat{p}$? ( Il y a une solution facile sans trop de calculation. )
\item Quelle est la valeur prévue d'energie? $\hat{H}$, utilisez $E_{1}$ et $E_{2}$ dans la solution. Les energies sont celles qui correspondent
aux fontions $\Psi_{1}$ et $\Psi_{2}$.
\end{enumerate}
\section{Exercise}
\par Une particule de masse $m$ est dans c'état
\begin{equation*}
\Psi(x,t) = Ae^{-a[\frac{mx^{2}}{\hbar} + it]}
\end{equation*}
où A et a sont des constantes positives. Obtenez la valeur de A! Quelle est la potentielle avec qui la fonctionne $\Psi(x,t)$ résoude l'équation
Schrödinger dans le cas unidimensionnelle? La variance d'un opérateur est défini comme:
\begin{equation*}
\sigma_{A} := \sqrt{<A^{2}> - <A>^{2}}
\end{equation*}
Présentez que la solution remplis la relation Heisenberg:
\begin{equation*}
\sigma_{\hat{x}}\sigma_{\hat{p}} \geq \frac{\hbar}{2}
\end{equation*}
\section{Exercise}
\par Une particule de masse $m$ viens de $+\infty$ dans la potentielle
\begin{equation*}
V(x) := \begin{cases}
	0 & \text{0 $\leq$ $x$} \\
	-V_{0} & \text{ $x$ $<$ $0$ } \\
\end{cases}
\end{equation*}
La particule disperse sur la potentielle donc $E > 0$ et $V_{0} > 0$. Quelle est la probabilité de transmission et de reflection? Obtenez les valuers
du coefficient de transmission (T) et celle de reflection (R)!
\section{Exercise}
\par Obtenez le spectre de le perticule dans la potentielle suivante
\begin{equation*}
V(x) := \begin{cases}
	V_{0} & n(a+b) < x < n(a+b) + a \\
	0 & \text{autrement}  \\
\end{cases}
\end{equation*}
$n \in Z$. (Aide: il y a une invariance de translation dans la potentielle donc les probabilités sont invariantes de celle)
\end{document}